\documentclass{article}

\usepackage[utf8]{inputenc}
\usepackage{polski}
\usepackage{indentfirst} 
\usepackage[margin=2.5cm]{geometry}
\usepackage[T1]{fontenc}
\usepackage{dejavu}
\usepackage{textcomp}
\usepackage{mathtools}

\newcommand\bitem[1]{\item[\textbullet~#1]}

\title{Projekt Architektura Komputerów MIPS}
\author{Wiktor Ślęczka}

\date{\today}
\begin{document}
\maketitle

\begin{abstract}
Program ma obliczyć odległość Hamminga pomiędzy dwoma 1-bitowymi bitmapami o rozdzielczości~64x64, uwzględniając możliwość
przesunięcia o 7 pixeli w każdą stronę.
\end{abstract}

\section{Program}
\subsection{Opis}
Program liczy minimalną odległość Hamminga pomiędzy dwoma bitmapami. Zgodnie ze specyfikacją, bitmapa powinna być 1bitowa, oraz
mieć rodzielczość 64x64. Ogległość należy policzyć dla każdego możliwego przesunięcia z zakresu <-7, 7> pixeli, zarówno poziomo jak
i pionowo\footnote{Łącznie to 225 kombinacji}. 
\subsection{Wejście}
Program wczytuje dwa pliki, 'obraz1.bmp' oraz 'obraz2.bmp', zawierające poprawne 1bitowe bitmapy o rozdzielczości 64x64.
\subsection{Wyjście}
Program powinien stworzyć dwa pliki:
\begin{description}
 \bitem{hamming.txt} w którym znajduje się najmniejsza znaleziona wartość.
 \bitem{tablica.txt} w którym znajdują się wszystkie obliczone wartości.
\end{description}
lub wypisać odpowiedni komunikat błądu na konsolę.

\section{Struktura programu}
Program zawiera następujące procedury:
\begin{description}
 \bitem{main} główna funkcja programu
 \bitem{counter} oblicza odległość Hamminga poprzez zawężanie zakresu do kolejno rzędu i półsłowa\footnote{Oryginalnie 'halfword'}, 
 a następnie zastosowanie w pętli operacji XOR dla wszystkich kombinacji i przesunięć bitów zawartych we wczytanym półsłowie. Operuje się tu 
 na pojedyńczych bajtach.
 \bitem{popcount} algorytm zliczania jedynek, wykonywany w wersji dla 32 bitów. Przy pomocy odpowiednich przesunięć i masek
 każda jedynka traktowana jest jako swój własny licznik, a następnie dodawana do sąsiednich.
 \bitem{minimal} znajduje najmniejszą wartość wśród obliczonych.
 \bitem{read} wczytuje plik o podanej nazwie.
 \bitem{write} zapisuje dany ciąg znaków do pliku.
\end{description}

\section{Struktury danych}
\begin{description}
 \bitem{array} tablica z wynikami obliczeń
 \bitem{prints} tablica z wynikami obliczeń przekonwertowanymi na ASCII.
 \bitem{file<*>} miejsce na wczytywane pliki.
 \bitem{mask<*>} używane w programie maski binarne.
\end{description}

\section{Algorytmy}
\subsection{Procedura counter}
Procedura counter służy obliczeniu odległości Hamminga. Na początek wyszczególnia dwa wiersze, następnie dla każdego bitu poza ostatnim  
wczytuje kolejne dwa bity w odwrotnej kolejność niż w pamięci\footnote{Ponieważ bajty są przechowywane w postaci Little-Endian.}.
Następnie otrzymane półsłowa są przesuwane i obcinane do oktetów bitów, i za pomocą operacji XOR obliczana jest odległość Hamminga dla
poszczególnych kombinacji oktetów. Powtarzane jest to odpowiednią ilość razy dla kolejnych wierszy, aby otrzymać przesunięcie w pionie.
Obliczone wartości dodawane są do odpowiedniego pola w tabeli array.
\subsection{Procedura popcount}
Procedura popcount służy do zliczania jedynek. Polega ona na trzech prostych krokach. Najpierw należy przesunąć wartość rejestru, który 
zliczamy o $2^{N-1}$, gdzie N to numer kroku, a następnie zastosować dla obu wartości AND z maską o sicie $2^{N-1}$ i dodać je do siebie.
Należy powtarzać te kroki aż jeden z rejestrów w pierwszym kroku zostanie wyzerowany.
W ten sposób każdy bit jest traktowany jako swój licznik, który następnie zlicza 2, 4, 8.. itd. bitów.
\section{Testy}
W katalogach test<n> znajdują się testy mające sprawdzić skrajne i klasyczne działanie programu. Są to odpowiednio:
\begin{enumerate}
 \item Dwie białe plansze
 \item Biała plansza z czarną kropką 4x4 w lewym górnym rogu i czysta biała plansza.
 \item Biała i czarna plansza.
 \item Biała plansza oraz biała plansza z czarną kropką pośrodku.
 \item Litera A na białym tle przesunięta względem drugiego obrazka o [5, -2].
\end{enumerate}
W katalogach znajdują się również pliki o rozszerzeniu .res, zawierające prawidłowe rozwiązania.
\end{document}
